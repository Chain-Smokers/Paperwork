\documentclass[journal,onecolumn]{IEEEtran}
\usepackage{graphicx}
\usepackage{multirow}

\renewcommand\IEEEkeywordsname{Keywords}

\title{Implementation Paper}
\author{
  Arvind Sudarshan,
  \and
  Chatane Shree,
  \and
  Eksambekar Yash,
  \and
  Gadkari Gaurav
}

\begin{document}

  \maketitle

  \begin{abstract}
    Voting in democratic country is a fundamental right granted to every eligible individual by the constitution. Current e-Voting system used isn’t transparent and can be improved in a few aspects. All voting data from Electronic Voting Machines (EVMs) are stored on a central server. This creates a single point of failure which can be exploited and tampered with easily. Such flaws cause mistrust in the electoral process. Blockchain is a shared  immutable ledger that facilitates the process of recording transactions in a network. It is an emerging technology whose full potential is yet to be realized. Blockchain became popular in 2009 when bitcoin was introduced and used as an alternative to tangible currency and has evolved since. It is a reliable system that can be used in various critical industrial applications. Blockchain has potential to improve the voting system to contest transparent and fair voting. Using this modern technology, a voting system can be implemented which provides transparency leading to fairness in the system. Furthermore, this will overcome the current system flaw of having a single point of failure caused by storing data in a centralized server. In addition to this, election results can be declared faster compared to the current system which might take a few days. The proposed system in this paper shows implementation of voting using blockchain technology.
  \end{abstract}
  \begin{IEEEkeywords}
  \end{IEEEkeywords}

  \section{Introduction}

  \section{Literature Survey}
  \begin{enumerate}
  	\item
  	  \textbf{Paper Name:} A Framework to Make Voting System Transparent Using Blockchain Technology\\
	  \textbf{Authors:} M. S. Farooq, U. Iftikhar and A. Khelifi\\
	  \textbf{Abstract:} A widespread mistrust towards the traditional voting system has made democratic voting in any country very critical. People have seen their fundamental rights being violated. Other digital voting systems have been challenged due to a lack of transparency. Most voting systems are not transparent enough; this makes it very difficult for the government to gain voters’ trust. The reason behind the failure of the traditional and current digital voting system is that it can be easily exploited. The primary objective is to resolve problems of the traditional and digital voting system, which include any kind of mishap or injustice during the process of voting. Blockchain technology can be used in the voting system to have a fair election and reduce injustice. The physical voting systems have many flaws in it as well as the digital voting systems are not perfect enough to be implemented on large scale. This appraises the need for a solution to secure the democratic rights of the people. This article presents a platform based on modern technology blockchain that provides maximum transparency and reliability of the system to build a trustful relationship between voters and election authorities. The proposed platform provides a framework that can be implemented to conduct voting activity digitally through blockchain without involving any physical polling stations. Our proposed framework supports a scalable blockchain, by using flexible consensus algorithms. The Chain Security Algorithm applied in the voting system makes the voting transaction more secure. Smart contracts provide a secure connection between the user and the network while executing a transaction in the chain. The security of the blockchain based voting system has also been discussed. Additionally, encryption of transactions using cryptographic hash and prevention of attack 51\% on the blockchain has also been elaborated. Furthermore, the methodology for carrying out blockchain transactions during the process of voting has been elaborated using Blockchain Finally, the performance evaluation of the proposed system shows that the system can be implemented in a large-scale population.
	\item
  	  \textbf{Paper Name:} Analysis of Blockchain Solutions for E-Voting: A Systematic Literature Review\\
	  \textbf{Authors:} A. Benabdallah, A. Audras, L. Coudert, N. El Madhoun and M. Badra\\
	  \textbf{Abstract:} To this day, abstention rates continue to rise, largely due to the need to travel to vote. This is why remote e-voting will increase the turnout by allowing everyone to vote without the need to travel. It will also minimize the risks and obtain results in a faster way compared to a traditional vote with paper ballots. In fact, given the high stakes of an election, a remote e-voting solution must meet the highest standards of security, reliability, and transparency to gain the trust of citizens. In literature, several remote e-voting solutions based on blockchain technology have been proposed. Indeed, the blockchain technology is proposed today as a new technical infrastructure for several types of IT applications because it allows to remove the TTP and decentralize transactions while offering a transparent and fully protected data storage. In addition, it allows to implement in its environment the smart-contracts technology which is used to automate and execute agreements between users. In this paper, we are interested in reviewing the most revealing e-voting solutions based on blockchain technology.
	\item
  	  \textbf{Paper Name:} BlockVOTE : An Architecture of a Blockchain-based Electronic Voting System\\
	  \textbf{Authors:} C. Angsuchotmetee, P. Setthawong and S. Udomviriyalanon\\
	  \textbf{Abstract:} Electronic voting systems provide many advantages over traditional ballot based voting systems mainly over the accuracy and speed of the tallying process of the voting. However, electronic voting systems suffer from many technical and security issues which have limited its deployment in voting scenarios such as company voting and political elections. Centralized electronic voting systems are, by nature not secure, and there are many avenues of cyber-attacks that could tamper the voting result. Electronic voting system should be highly secured, tamperedproof guaranteed, and the voting should be trusted worthy. In this study, we propose BlockVOTE, a Blockchain-based electronic voting system. Our proposal uses Blockchain to ensure that the voting process can be kept secure and trustable through the consensus handling mechanism of the Blockchain. The architecture design and implementation suggestion are provided in this study. The implementation of the proposal was developed and tested via experimentation. The experiment result and the discussion on the possibility of adopting our proposal in an actual election is provided at the end of this study.
  \end{enumerate}

  \section{Proposed System}
  
  \section{Design and Implementation}
  
  \section{Performance Evaluation}

  \section{Conclusion}

  \bibliographystyle{ieeetr}
  \bibliography{References}

\end{document}