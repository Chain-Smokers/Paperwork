\documentclass{article}
\usepackage{graphicx}
\usepackage{titling}
\graphicspath{ {../Resources/} }
\usepackage{babel}

\title{\includegraphics[width=\textwidth]{letterhead.png}\\{\large\textbf{DEPARTMENT OF COMPUTER ENGINEERING}}
\\
\ \\
\large{\textbf{Project Title :} Decentralized and Secure Voting System using Blockchain Technology}}
\posttitle{\par\end{center}}
\date{{\vspace{-8ex}}}
\pagenumbering{gobble}

\begin{document}

\maketitle

\section*{Abstract}
Voting in democratic country is a fundamental right granted to every eligible individual by the constitution. Current e-Voting system used isn't transparent and can be improved in a few aspects. All voting data from Electronic Voting Machines (EVMs) are stored on a central server. This creates a single point of failure which can be exploited and tampered with easily. Such flaws cause mistrust in the electoral process. Blockchain is a shared  immutable ledger that facilitates the process of recording transactions in a network. It is an emerging technology whose full potential is yet to be realized. Blockchain became popular in 2009 when bitcoin was introduced and used as an alternative to tangible currency and has evolved since. It is a reliable system that can be used in various critical industrial applications. Blockchain has potential to improve the voting system to contest transparent and fair voting. Using this modern technology, a voting system can be implemented which provides transparency leading to fairness in the system. Furthermore, this will overcome the current system flaw of having a single point of failure caused by storing data in a centralized server. In addition to this, election results can be declared faster compared to the current system which might take a few days. The proposed system in this paper shows implementation of voting using blockchain technology.
\\\\
\textbf{Keywords :} Blockchain Technology, Ethereum Virtual Machine, Voting System, Secure Voting
\\\\\\
\begin{tabular}{ p{20em} c }
    \textbf{Project Group Members}  & \textbf{Guide}    \\
    Arvind Sudarshan (19CO006)                          \\
    Chatane Shree Atul (19CO011)                        \\
    Eksambekar Yash Sagar (19CO020)                     \\
    Gadkari Gaurav Sudhir (19CO022) & Prof. S. S. Kolte
\end{tabular}



\end{document}